\documentclass[10pt,a4paper]{report}
\usepackage[utf8]{inputenc}
\usepackage[english]{babel}
\usepackage[T1]{fontenc}
\usepackage{amsmath}
\usepackage{amsfonts}
\usepackage{amssymb}
\usepackage{makeidx}
\usepackage[pdftex]{graphicx}
\usepackage{lmodern}
\usepackage{kpfonts}
%\usepackage{fourier}
\usepackage{multicol,caption}
\usepackage{float}
\usepackage{listings}
\usepackage{xcolor}
\usepackage{ifthen}
\usepackage{etoolbox}
\usepackage{lastpage}
\usepackage{fancyhdr}
\usepackage{lipsum}
\usepackage{appendix}
\usepackage[left=2cm,right=2cm,top=3cm,bottom=4cm, headsep=55pt, , inner=2.5cm, outer=1.5cm]{geometry}
\usepackage[skip=10pt plus1pt, indent=0pt]{parskip}
\usepackage{array}
\usepackage{tabularx}
\usepackage{tabularray}
\usepackage{tocloft}
\usepackage{tikz}
\usetikzlibrary{positioning}
\usepackage{tikz-cd}
\usetikzlibrary{fit,shapes.geometric}

\usepackage{calrsfs}
\DeclareMathAlphabet{\pazocal}{OMS}{zplm}{m}{n}


\tikzset{
   actor/.style={
     circle,
     draw=black,
     minimum size=3mm
   },
   arrow/.style={
     -latex,
     thick,
     shorten <=2pt,
     shorten >=2pt
   },
   column/.style={
   -latex,
   sep=-6ex, 
   row sep=1ex
   }
}

%%gmedina solution
\newcommand{\listequationsname}{List of Equations}
\newlistof{myequations}{equ}{\listequationsname}
\newcommand{\myequations}[1]{%
\addcontentsline{equ}{myequations}{\protect\numberline{\theequation}#1}\par}
\setlength{\cftmyequationsnumwidth}{2.5em}% Width of equation number in List of Equations


\newcommand{\listremarksname}{List of Remarks}
\newlistof{myremarks}{rmk}{\listremarksname}
\newcounter{rmkcntr}
\newcommand{\myremarks}[2]{%
\stepcounter{rmkcntr}%
\addcontentsline{rmk}{myremarks}{\protect\numberline{\thermkcntr}#1}\par\fbox{%
\begin{minipage}[c]{0.3\textwidth}
\textit{\textbf{Remark \thermkcntr: #1 }}
\end{minipage}%
\vrule\enskip\vrule\quad
\begin{minipage}{\dimexpr 0.7\textwidth-0.8pt-1.5em}
\textit{#2} 
%\label{rq:rq0}
\end{minipage} \\
}}
\setlength{\cftmyremarksnumwidth}{2.5em}% Width of remark number in List of remarks



\newcommand{\listdefinitionsname}{List of definitions}
\newlistof{mydefinitions}{def}{\listdefinitionsname}
\newcounter{defcntr}
\newcommand{\mydefinitions}[2]{%
\stepcounter{defcntr}%
\addcontentsline{def}{mydefinitions}{\protect\numberline{\thedefcntr}#1}\par\fbox{%
\begin{minipage}[c]{0.3\textwidth}
\textit{\textbf{Definition \thedefcntr: #1 }}
\end{minipage}%
\vrule\enskip\vrule\quad
\begin{minipage}{\dimexpr 0.7\textwidth-0.8pt-1.5em}
\textit{#2} 
%\label{rq:rq0}
\end{minipage} \\
}}
\setlength{\cftmydefinitionsnumwidth}{2.5em}% Width of definition number in List of definitions



\pagestyle{headings}
% definition paragaph en subsubsubsection
\setcounter{secnumdepth}{5}
\setcounter{tocdepth}{5}
\makeatletter
\newcommand\subsubsubsection{\@startsection{paragraph}{4}{\z@}{-2.5ex\@plus -1ex \@minus -.25ex}{1.25ex \@plus .25ex}{\normalfont\normalsize\bfseries}}
\newcommand\subsubsubsubsection{\@startsection{subparagraph}{5}{\z@}{-2.5ex\@plus -1ex \@minus -.25ex}{1.25ex \@plus .25ex}{\normalfont\normalsize\bfseries}}
\makeatother

\definecolor{codegreen}{rgb}{0,0.6,0}
\definecolor{codegray}{rgb}{0.5,0.5,0.5}
\definecolor{codepurple}{rgb}{0.58,0,0.82}
\definecolor{backcolour}{rgb}{0.95,0.95,0.92}

\lstdefinestyle{mystyle}{
    backgroundcolor=\color{backcolour},   
    commentstyle=\color{codegreen},
    keywordstyle=\color{magenta},
    numberstyle=\tiny\color{codegray},
    stringstyle=\color{codepurple},
    basicstyle=\ttfamily\fontsize{8pt}{8pt}\selectfont,
    breakatwhitespace=false,         
    breaklines=true,                 
    captionpos=b,                    
    keepspaces=true,                 
    numbers=left,                    
    numbersep=5pt,                  
    showspaces=false,                
    showstringspaces=false,
    showtabs=false,                  
    tabsize=5
}

\fancypagestyle{plain}{
\fancyhf{}% Clear header/footer
\renewcommand{\headrulewidth}{0pt}
\fancyhead[LO,LE]{\includegraphics[width=1cm]{Logo-golfball.jpg}\hrule}
\fancyhead[RO,RE]{ \textbf{Golf Simulator in Scilab} v0.1 \\}
\fancyhead[CO,CE]{\textcolor{red}{free}}
\fancyfoot[LO,CE]{\fontsize{8}{11}\selectfont\textit{\begin{center}This document and the information contained herein are the property of Matthieu PHILIPPE.\\They must not be copied or disclosed to third parties without prior written authorization.\end{center}}\begin{flushright}-M\end{flushright}}
}

\renewcommand{\chaptermark}[1]{\markboth{#1}{#1}}

\fancypagestyle{plain2}{
\fancyhf{}% Clear header/footer
\renewcommand{\headrulewidth}{0pt}
\fancyhead[LO,LE]{\includegraphics[width=1cm]{Logo-golfball.jpg}\hrule}
\fancyhead[RO,RE]{ \textbf{\chaptername\ \thechapter\ --\ \leftmark\ }\\Page \thepage\  / \pageref{LastPage} }
\fancyhead[CO,CE]{\textcolor{red}{free}}
\fancyfoot[LO,CE]{\fontsize{8}{11}\selectfont\textit{\begin{center}This document and the information contained herein are the property of Matthieu PHILIPPE.\\They must not be copied or disclosed to third parties without prior written authorization.\end{center}}\begin{flushright}-M\end{flushright}}
}

\fancypagestyle{plainstar}{
\fancyhf{}% Clear header/footer
\renewcommand{\headrulewidth}{0pt}
\fancyhead[LO,LE]{\includegraphics[width=1cm]{Logo-golfball.jpg}\hrule}
\fancyhead[RO,RE]{ \textbf{\ }\\Page \thepage\  / \pageref{LastPage} }
\fancyhead[CO,CE]{\textcolor{red}{free}}
\fancyfoot[LO,CE]{\fontsize{8}{11}\selectfont\textit{\begin{center}This document and the information contained herein are the property of Matthieu PHILIPPE.\\They must not be copied or disclosed to third parties without prior written authorization.\end{center}}\begin{flushright}-M\end{flushright}}
}



\fancypagestyle{noheader}{
\fancyhf{}% Clear header/footer
\fancyhead[LO,LE]{}
\fancyhead[CO,CE]{\textcolor{red}{free}}
\fancyhead[RO,RE]{}
}

\lstset{style=mystyle}

\usepackage[left=2cm,right=2cm,top=3cm,bottom=4cm, headsep=55pt]{geometry}

\author{Matthieu PHILIPPE\\Publication Perso}

\title{\textbf{Golf Simulator in Scilab:\\Capturing a swing and estimating the cary\\with a tiny low-cost  Radar}}

\begin{document}
\sffamily
\pagestyle{plain}

\rmfamily\maketitle\sffamily
%\noindent\makebox[\linewidth]{\rule{\textwidth}{0.4pt}}
\tableofcontents
\listoffigures
\listoftables
\lstlistoflistings
\listofmyequations
\listofmyremarks
\listofmydefinitions

\newpage
\pagestyle{plainstar}
\section*{ABSTRACT}\setcounter{page}{1}
By using the equations of golf ball flight dynamics, this article demonstrates how to code a swing simulator for a ball in Scilab.

The simulator uses an ODE (ordinary differential equation) solver from Scilab, the fluid dynamics equations on a ball (taking into account gravity, drag forces, and Magnus forces), the restitution of forces from a golf club applied to a ball for launches, its  flight, bounce and roll. 

In the second part, the article presents a method for capturing the speed of the ball in flight using 2 low-cost radars, coupled with an Arduino Mini and an adapted sound card. 

The radar signal is recorded in WAV format and filtered to extract speed and spin information during flight, using spectral density.

\emph{\textbf{Keywords:} Golf Ball, Simulator, Scilab, Radar, Spectral dentisty.}

\chapter{Introduction}
After spending a few years playing golf, I wanted to learn more about the dynamics of the interactions between the club, the ball, and the elements. 

Of course, the academic literature and publications were quite abundant. But I came accross articles \cite{PWBJKH76} written in 1976 and \cite{SASDR94} in 1994. The description of the ball's physics pushed me to undertake modeling this system in Scilab. Subsequently, I did it in Java, C++, and then C\# to use it under \emph{Unity game engine}{\textregistered \texttrademark}.

When the results were satisfactory, I thought about capturing the flight of a golf ball with instrumentation. After tests with sonar, which were largely insufficient, I took small presence detection radars for automatic doors.

I didn't know what I was getting into, but after a updating my knowledge on signal filtering and a few lines of programming in Scilab, I found a way to capture the essential data. Namely the speed and backspin of the flight of a golf ball.

This article describes all items I found, and how I set all them together. Equations, programs, filtering, estimators and electronics compounents to make a Virtual Golf Simulation.

All the equations presented in this article are derived from public publications or books, with references listed in the bibliography. I will not revisit the demonstrations and physical explanations, as they are not the focus of these pages. 
What matters to me is the relationship between this information and how I have used or adapted it.

\chapter{Golf Ball Flight}

The main equation is the ball's flight. The equation accounts 3 forces : Magnus Force, Air Drag Force and gravity, such as :

\begin{equation}\label{eq:flightforces}
\sum{\overrightarrow{F}} =  \overrightarrow{F_{m}} + \overrightarrow{F_{d}} + \overrightarrow{F_{g}}
\end{equation}\myequations{Flight Forces}

\section{Drag Force}
\subsection{Air characteristics}

\mydefinitions{The Air Drag Force}{The Air Drag Force is the resitance exposed on the cross-sectional area of an object, orthogonal to the displacement, determined by the relation given by \cite{BBRS11} as \ref{eq:aircaracteristic2}} :

\begin{eqnarray}
F_d(V)&=&-\frac{1}{2}.\rho(T).C_d.S.V^2(t)\label{eq:aircaracteristic2}
\end{eqnarray}\myequations{Air Force Resistance}

where:
\begin{eqnarray*}\label{eq:aircaracteristiccst}
\rho(T)&=&1,292*\frac{273,15}{273,15+ T}\label{eq:aircaracteristic1}\\
\rho(T)\ &=&Air\ density\ Rho\ depending\ on\ temperature\ T\\
T&=&Temperature\\
C_d\ &=&Drag\ Coefficient,\ depending\ on\ the\ object\\
V&=&Object's\ Velocity\\
F_d(V)\ &=&Drag\ Force\ due\ to\ Air\ Resistance\ depending\ on\ velocity\ V\\
S&=& The\ cross-sectional\ area\ exposed,\ orthogonal\ to\ the\ displacement.
\end{eqnarray*}

\subsection{Ball's characteristics in the air}

Considering those characteristics of a dimpled ball model,

\begin{eqnarray*}\label{eq:massdiameterintertia}
mass\ m&=&0,04593\ g \\
Dynamic\ viscosity\ \mu&=&1,5.10^{-6} m^2/s\\
Diameter\ D&=&0,04267\ m\\
Radius\ r&=&\frac{D}{2}\\
Inertia\ I&=&\frac{2}{5}.m.R^2\\
Cross-section\  S&=&\pi.R^2\\
Ball\ Velocity\ &=&V
\end{eqnarray*}\myequations{Ball Mass, Diameter, Rayon, Intertia, Section, Reyolds Number}

We have an estimated models,

\begin{eqnarray}\label{eq:dragforce}
Re&=&\frac{\rho(T).V.D}{\mu}\label{eq:reynoldsnb}\\
C_d&=&0,36 + \frac{24}{Re} + \frac{6}{1+\sqrt{Re}}\\
\overrightarrow{F_{d}} &=& -\frac{1}{2}.\rho(T).C_d.S.\sqrt{(V(x)^2+V(y)^2+V(z)^2)}.\widehat{V}
\end{eqnarray}\myequations{Drag Coefficient, Drag Force}

where,

\begin{eqnarray*}\label{eq:dragforce}
\widehat{V}&=&Unit\ vector\ of\ the\ velocity\\
Re&=&Reynolds\ Number\ estimator\\
C_d&=&Drag\ Coefficient\ White's\ relation\\
\overrightarrow{F_{d}}&=& Drag\ Force
\end{eqnarray*}


\myremarks{Finally, the equation of the drag Force along the 3-axes becomes \ref{eq:vectordragforce}}{\begin{equation}\label{eq:vectordragforce}
\overrightarrow{F_d} = \begin{pmatrix}
-\frac{1}{2}.\rho(T).C_d.S.V_{x}^2 \\
-\frac{1}{2}.\rho(T).C_d.S.V_{y}^2 \\
-\frac{1}{2}.\rho(T).C_d.S.V_{z}^2
\end{pmatrix}
\end{equation}
}

\section{Magnus Effect}
\mydefinitions{The Magnus Effect}{The Magnus \cite{magnus1853abweichung} effect is due to the spin of a cylindrical object, moving in a fluid (air). It is called the Robin effect for a ball.} 

But by generalization, the Magnus effect name is applied to any type of object subjected to this suction force. This lift force occurs, altering the ballistic trajectory of the golf ball. The Force depends on the Lift coefficient (\emph{$C_l$}), that is associated with the object, according to its shape, nature and texture. So that the force is written as a composant of Object Velocity, SpinRate and a coefficient of Magnus effect \emph{$C_m$} (\ref{eq:magnusforce1}),


\begin{eqnarray}\label{eq:magnusforce}
\overrightarrow{F_{m}} &=& C_m.(\overrightarrow{\omega}\ \times\  \overrightarrow{V})\label{eq:magnusforce1}
\end{eqnarray}\myequations{Magnus Lift effect}


where,
\begin{eqnarray*}\label{eq:dragforce}
\omega&=&spin\ in\ rad/s\\
V_p&=&r\omega\ periferal\ velocity\\
V&=&Ball\  Velocity\\
C_m&=&Coefficient\ of\ Magnus\ effect
\end{eqnarray*}

The main bibliography for golf ball I used, is \cite{GPA05} and \cite{SASDR94}. I have found several ways to calculate the $C_m$, given in expressions bellow.  The first simple aproximation is given by \ref{eq:magnusforce2}, and the Coefficient of Magnus effect can be given by the first two estimators bellow \ref{eq:magnusforce3}, \ref{eq:magnusforce4} (\cite{GPA05}). But I kept this one \ref{eq:magnusforce5} (\cite{SASDR94},  \cite{AL98}). 

\begin{eqnarray}\label{eq:magnusforce0}
F_m&=& \frac{1}{2}.C_m.\rho.\mu^2\label{eq:magnusforce2}\\
C_m&=&\frac{r\omega}{V}\label{eq:magnusforce3}\\
C_m&=&-0,05+ \sqrt{0,0025 + 0,36\left( \frac{r\omega}{V}\right)}\label{eq:magnusforce4}\\
C_m &=&0,5 . \rho(T). S. R^2. C_l.\mid{\frac{\omega}{V}}\mid^{0,4}.V\label{eq:magnusforce5}\\
C_l&=&\frac{1}{2 + (V/R\omega)*sin(\zeta)}\label{eq:magnusforce6}
\end{eqnarray}\myequations{Magnus effect and different coefficient models}

\cite{GGA} gives a correction or adjustment coefficient $C_{l}$ (\ref{eq:magnusforce6}) to (\ref{eq:magnusforce5}), where $\zeta$ in Lift Coefficient, is the angle between the axis of rotation and the direction of motion. If R$\omega\ll$ V, then $C_l \simeq R\omega \frac{sin\zeta}{V}$. Certainly more appropriate than (\ref{eq:magnusforce3}) to take into account an effect of side spin and backspin.

\begin{eqnarray}\label{eq:zeta}
sin\zeta = \frac{\parallel\overrightarrow{\omega} \times \overrightarrow{V}\parallel}{\omega.V} = \frac{1}{\omega.V}.\sqrt{((\omega_j.V_k -\omega_k.V_j )^2 + ( \omega_k.V_i -\omega_i.V_k )^2 +( \omega_i.V_j -\omega_j.V_i )^2)}
\end{eqnarray}\myequations{Angle between mouvment and spin axes}

In order to meet my needs, I have choosen to provide my adapted $C_l$ values to (\ref{eq:magnusforce5}) instead of (\ref{eq:magnusforce6}), from tests with my own balls (texture and  number of dipples impact Reynolds number (\ref{eq:reynoldsnb}) as shown in \cite{ASCM11}). These data are recorded in a Scilab table $C_{l1}$. $C_m$ becomes (\ref{eq:magnusforce7}). 
\begin{eqnarray}
C_m &=&0,5 . \rho(T). S. R^2. C_{l1}.\mid{\frac{\omega}{V}}\mid^{0,4}.V\label{eq:magnusforce7}
\end{eqnarray}
With my own Club Mizuno JPX and ball  Decthlon :
\begin{eqnarray}
C_{l1} &=&'D'  ;'W5'  ;'H3' ;'I5' ;'I6' ;'I7' ;'I8' ;'I9' ;'PW' ;'AW' ;'SW' ;'LW'\\
&&0.64;0.65;0.54;0.54;0.83;0.42;0.53;0.53;0.52;0.52;0.51;0.51\label{eq:mycl1}
\end{eqnarray}

\myremarks{Knowing \ref{eq:magnusforce1} and \ref{eq:magnusforce7} and \ref{eq:mycl1}, I chose the following Magnus force model}{\begin{equation}\label{eq:vectormagnusforce}
\overrightarrow{F_m} = \begin{pmatrix}
C_m*(\omega_j*v_k-\omega_k*v_j) \\
C_m*(\omega_k*v_i-\omega_i*v_k) \\
C_m*(\omega_i*v_j-\omega_j*v_i)
\end{pmatrix}
\end{equation}
}

\section{Gravity Force}
\mydefinitions{The gravity force}{The gravity force is determined by the acceleration \emph{g} to the ground.}

So that, the expression is simply,
\myremarks{The Gravity Force}{\begin{equation}\label{eq:vectorgravityforce}
\overrightarrow{F_g} = \begin{pmatrix}
0\\
-mg\\
0
 \end{pmatrix}
\end{equation}\myequations{Gravity Force}
}


\section{Launch Velocity}
Due to the impact of the golf club to the ball, it results a ball flight with a launch angle. The club head loft determines the launch angle, the club head speed and the texture of both determine the launch ball speed and ball spin \cite{PA01}.

Due to the texture we have a loss in energy, that can be taken into account throught the Coefficient Of Restitution \emph{e} (COR \cite{BLSJ94}).

\begin{figure} [H]\centering
\includegraphics[width=6cm,keepaspectratio, ]{ImpactClubBall.png}
\caption{Velocity Components}
\label{fig:impactclubball}
\end{figure}


\cite{PA01} gives an estimation of COR e : 
\begin{eqnarray}
e&=&0,86\ -\ 0,0029.V_{ci}.cos(\theta)\\
V_{bfn}&=&V_{ci}.cos(\theta).\frac{1+e}{1+\frac{m}{M}}\\
V_{bfp}&=&-V_{ci}.\frac{sin(\theta)}{1+\frac{m}{M} + \frac{5}{2}}\\
\psi_{bo}&=&\theta + tan^{-1}(\frac{V_{bfp}}{V_{bfn}})\\
\omega_{bo}&=&\omega_{bf} = -m.V_{bfp}.\frac{R}{I}
\end{eqnarray}\myequations{Ball velocity and launch angle after the impact of the golf club}


where,
\begin{eqnarray*}
V_{ci}&=&Initial\ Club\ Velocity,\ parallel\ to\ the\ ground.\\
V_{bfn}&=&Final\ Ball\ Velocity\ Orthogonal\ to\ golf\ club\ face\\
V_{bfp}&=&Velocity\ of\ the\ golf\ ball,\ parallel\ of\ the\ club\ face\\
\theta&=&dynamic\ loft\ (\ club\ loft + Shaft\ Lean\ Impact\ <\  loft)\\
e&=&COR,\ e\simeq0.78\ for\ Irons,\ e\simeq0.83\ for\ Drives\ and\ Woods\\
\psi_{bo}&=&Launch\ Angle\\
\omega_{bo}&=&Spin\ of\ the\ glof\ ball\\
I&=&Inertia\ (\ref{eq:massdiameterintertia})
\end{eqnarray*}

\section{Miss the sweet spot}
It is difficult to perfectly center the ball and the club. The ball velocity model is modified to include a \emph{Miss} parameter. It is the amount by which the sweet spot is missed. The initial ball velocity is (\ref{eq.miss}),
\begin{equation}\label{eq.miss}
V_{bi}=(1-35.56*miss).\sqrt{V_{bfn}^2+V_{bfp}^2}\ (miss\ mesured\ in\ meter)
\end{equation}\myequations{Intial ball velocity after missed the sweet spot}

\section{Club Face Initial Velocity}
Due to the Club Path ($\alpha_{cp}$) and the Face to Path and Face Angle ($\gamma_{fp}$), the initial ball velocity and spin is modified.

\begin{figure} [H]\centering
\includegraphics[width=12cm,keepaspectratio, ]{clubpath.png}
\caption{Club Path}
\label{fig:clubpath}
\end{figure}

\section{Complete Model of flight}

\section{Ball's characteristics bouncing on the grass}
\cite{rwlc2010}
\section{Ball's characteristics rolling on the grass}
\cite{PA07}
\cite{PA022}


\chapter{Golf Ball Capturing}
\section{section}
section
\subsection{subsection}
subsection

\chapter{Listings}
\section{Golfball.sci}
\begin{lstlisting}[language=matlab, caption=Listing of the golf ball fligth, label={lst:gaolfballsci}]
// Copyright (C) 2016 - Corporation - Author
//
// About your license if you have any
//
// Date of creation: 16 juin 2016
//
//SCI2C: DEFAULT_PRECISION= DOUBLE
//
// ReadWave : FFT_Mat.sce
// 

\end{lstlisting}

//\chapter*{}
\section*{Conflicts of Interest}
The authors declare no conflict of interest.

\bibliographystyle{alpha}
\selectlanguage{english}
\bibliography{biblioBibTex}
%\end{multicols}\newpage



%\input{PageEnd.tex}
\end{document}
