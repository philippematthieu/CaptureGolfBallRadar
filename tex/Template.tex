\documentclass[10pt,a4paper]{report}
\usepackage[utf8]{inputenc}
\usepackage[english]{babel}
\usepackage[T1]{fontenc}
\usepackage{amsmath}
\usepackage{amsfonts}
\usepackage{amssymb}
\usepackage{makeidx}
\usepackage[pdftex]{graphicx}
\usepackage{lmodern}
\usepackage{kpfonts}
%\usepackage{fourier}
\usepackage{multicol,caption}
\usepackage{float}
\usepackage{listings}
\usepackage{xcolor}
\usepackage{ifthen}
\usepackage{etoolbox}
\usepackage{lastpage}
\usepackage{fancyhdr}
\usepackage{lipsum}
\usepackage{appendix}
\usepackage[left=2cm,right=2cm,top=3cm,bottom=4cm, headsep=55pt, , inner=2.5cm, outer=1.5cm]{geometry}
\usepackage[skip=10pt plus1pt, indent=0pt]{parskip}
\usepackage{array}
\usepackage{tabularx}
\usepackage{tabularray}
\usepackage{tocloft}
\usepackage{tikz}
\usetikzlibrary{positioning}
\usepackage{tikz-cd}
\usetikzlibrary{fit,shapes.geometric}

\usepackage{calrsfs}
\DeclareMathAlphabet{\pazocal}{OMS}{zplm}{m}{n}


\tikzset{
   actor/.style={
     circle,
     draw=black,
     minimum size=3mm
   },
   arrow/.style={
     -latex,
     thick,
     shorten <=2pt,
     shorten >=2pt
   },
   column/.style={
   -latex,
   sep=-6ex, 
   row sep=1ex
   }
}

%%gmedina solution
\newcommand{\listequationsname}{List of Equations}
\newlistof{myequations}{equ}{\listequationsname}
\newcommand{\myequations}[1]{%
\addcontentsline{equ}{myequations}{\protect\numberline{\theequation}#1}\par}
\setlength{\cftmyequationsnumwidth}{2.5em}% Width of equation number in List of Equations


\newcommand{\listremarksname}{List of Remarks}
\newlistof{myremarks}{rmk}{\listremarksname}
\newcounter{rmkcntr}
\newcommand{\myremarks}[2]{%
\stepcounter{rmkcntr}%
\addcontentsline{rmk}{myremarks}{\protect\numberline{\thermkcntr}#1}\par\fbox{%
\begin{minipage}[c]{0.3\textwidth}
\textit{\textbf{Remark \thermkcntr: #1 }}
\end{minipage}%
\vrule\enskip\vrule\quad
\begin{minipage}{\dimexpr 0.7\textwidth-0.8pt-1.5em}
\textit{#2} 
%\label{rq:rq0}
\end{minipage} \\
}}
\setlength{\cftmyremarksnumwidth}{2.5em}% Width of remark number in List of remarks



\newcommand{\listdefinitionsname}{List of definitions}
\newlistof{mydefinitions}{def}{\listdefinitionsname}
\newcounter{defcntr}
\newcommand{\mydefinitions}[2]{%
\stepcounter{defcntr}%
\addcontentsline{def}{mydefinitions}{\protect\numberline{\thedefcntr}#1}\par\fbox{%
\begin{minipage}[c]{0.3\textwidth}
\textit{\textbf{definition \thedefcntr: #1 }}
\end{minipage}%
\vrule\enskip\vrule\quad
\begin{minipage}{\dimexpr 0.7\textwidth-0.8pt-1.5em}
\textit{#2} 
%\label{rq:rq0}
\end{minipage} \\
}}
\setlength{\cftmydefinitionsnumwidth}{2.5em}% Width of definition number in List of definitions



\pagestyle{headings}
% definition paragaph en subsubsubsection
\setcounter{secnumdepth}{5}
\setcounter{tocdepth}{5}
\makeatletter
\newcommand\subsubsubsection{\@startsection{paragraph}{4}{\z@}{-2.5ex\@plus -1ex \@minus -.25ex}{1.25ex \@plus .25ex}{\normalfont\normalsize\bfseries}}
\newcommand\subsubsubsubsection{\@startsection{subparagraph}{5}{\z@}{-2.5ex\@plus -1ex \@minus -.25ex}{1.25ex \@plus .25ex}{\normalfont\normalsize\bfseries}}
\makeatother

\definecolor{codegreen}{rgb}{0,0.6,0}
\definecolor{codegray}{rgb}{0.5,0.5,0.5}
\definecolor{codepurple}{rgb}{0.58,0,0.82}
\definecolor{backcolour}{rgb}{0.95,0.95,0.92}

\lstdefinestyle{mystyle}{
    backgroundcolor=\color{backcolour},   
    commentstyle=\color{codegreen},
    keywordstyle=\color{magenta},
    numberstyle=\tiny\color{codegray},
    stringstyle=\color{codepurple},
    basicstyle=\ttfamily\fontsize{8pt}{8pt}\selectfont,
    breakatwhitespace=false,         
    breaklines=true,                 
    captionpos=b,                    
    keepspaces=true,                 
    numbers=left,                    
    numbersep=5pt,                  
    showspaces=false,                
    showstringspaces=false,
    showtabs=false,                  
    tabsize=5
}

\fancypagestyle{plain}{
\fancyhf{}% Clear header/footer
\renewcommand{\headrulewidth}{0pt}
\fancyhead[LO,LE]{\includegraphics[width=1cm]{Logo-golfball.jpg}\hrule}
\fancyhead[RO,RE]{ \textbf{Golf Simulator in Scilab} v0.1 \\}
\fancyhead[CO,CE]{\textcolor{red}{free}}
\fancyfoot[LO,CE]{\fontsize{8}{11}\selectfont\textit{\begin{center}This document and the information contained herein are the property of Matthieu PHILIPPE.\\They must not be copied or disclosed to third parties without prior written authorization.\end{center}}\begin{flushright}-M\end{flushright}}
}

\renewcommand{\chaptermark}[1]{\markboth{#1}{#1}}

\fancypagestyle{plain2}{
\fancyhf{}% Clear header/footer
\renewcommand{\headrulewidth}{0pt}
\fancyhead[LO,LE]{\includegraphics[width=1cm]{Logo-golfball.jpg}\hrule}
\fancyhead[RO,RE]{ \textbf{\chaptername\ \thechapter\ --\ \leftmark\ }\\Page \thepage\  / \pageref{LastPage} }
\fancyhead[CO,CE]{\textcolor{red}{free}}
\fancyfoot[LO,CE]{\fontsize{8}{11}\selectfont\textit{\begin{center}This document and the information contained herein are the property of Matthieu PHILIPPE.\\They must not be copied or disclosed to third parties without prior written authorization.\end{center}}\begin{flushright}-M\end{flushright}}
}

\fancypagestyle{plainstar}{
\fancyhf{}% Clear header/footer
\renewcommand{\headrulewidth}{0pt}
\fancyhead[LO,LE]{\includegraphics[width=1cm]{Logo-golfball.jpg}\hrule}
\fancyhead[RO,RE]{ \textbf{\ }\\Page \thepage\  / \pageref{LastPage} }
\fancyhead[CO,CE]{\textcolor{red}{free}}
\fancyfoot[LO,CE]{\fontsize{8}{11}\selectfont\textit{\begin{center}This document and the information contained herein are the property of Matthieu PHILIPPE.\\They must not be copied or disclosed to third parties without prior written authorization.\end{center}}\begin{flushright}-M\end{flushright}}
}



\fancypagestyle{noheader}{
\fancyhf{}% Clear header/footer
\fancyhead[LO,LE]{}
\fancyhead[CO,CE]{\textcolor{red}{free}}
\fancyhead[RO,RE]{}
}

\lstset{style=mystyle}

\usepackage[left=2cm,right=2cm,top=3cm,bottom=4cm, headsep=55pt]{geometry}

\author{Matthieu PHILIPPE\\Publication Perso}

\title{\textbf{Golf Simulator in Scilab:\\Capturing a swing and estimating the cary\\with a tiny low-cost  Radar}}

\begin{document}
\sffamily
\pagestyle{plain}

\rmfamily\maketitle\sffamily
%\noindent\makebox[\linewidth]{\rule{\textwidth}{0.4pt}}
\tableofcontents
\listoffigures
\listoftables
\lstlistoflistings
\listofmyequations
\listofmyremarks
\listofmydefinitions

\newpage
\pagestyle{plainstar}
\section*{ABSTRACT}\setcounter{page}{1}
By using the equations of golf ball flight dynamics, this article demonstrates how to code a swing simulator for a ball in Scilab.

The simulator uses an ODE (ordinary differential equation) solver from Scilab, the fluid dynamics equations on a ball (taking into account gravity, drag forces, and Magnus forces), the restitution of forces from a golf club applied to a ball for launches, its  flight, bounce and roll. 

In the second part, the article presents a method for capturing the speed of the ball in flight using 2 low-cost radars, coupled with an Arduino Mini and an adapted sound card. 

The radar signal is recorded in WAV format and filtered to extract speed and spin information during flight, using spectral density.

\emph{\textbf{Keywords:} Golf Ball, Simulator, Scilab, Radar, Spectral dentisty.}

\chapter{Introduction}
After spending a few years playing golf, I wanted to learn more about the dynamics of the interactions between the club, the ball, and the elements. 

Of course, the academic literature and publications were quite abundant. But I came accross articles \cite{PWBJKH76} written in 1976 and \cite{SASDR94} in 1994. The description of the ball's physics pushed me to undertake modeling this system in Scilab. Subsequently, I did it in Java, C++, and then C\# to use it under \emph{Unity game engine}{\textregistered \texttrademark}.

When the results were satisfactory, I thought about capturing the flight of a golf ball with instrumentation. After tests with sonar, which were largely insufficient, I took small presence detection radars for automatic doors.

I didn't know what I was getting into, but after a updating my knowledge on signal filtering and a few lines of programming in Scilab, I found a way to capture the essential data. Namely the speed and backspin of the flight of a golf ball.

This article describes all items I found, and how I set all them together. Equations, programs, filtering, estimators and electronics compounents to make a Virtual Golf Simulation.

All the equations presented in this article are derived from public publications or books, with references listed in the bibliography. I will not revisit the demonstrations and physical explanations, as they are not the focus of these pages. 
What matters to me is the relationship between this information and how I have used or adapted it.

\chapter{Golf Ball Flight}

The main equation is the ball's flight. The equation accounts 3 forces : Magnus Force, Air Drag Force and gravity, such as :

\begin{equation}\label{eq:flightforces}
\sum{F} =  F_{M} - m.g + F_{d}
\end{equation}\myequations{Flight Forces}

\section{Drag Force}
\subsection{Air characteristics}

The Air Drag Force resitance is determined by the relation given by \cite{BBRS11} as \ref{eq:aircaracteristic2} :

\begin{eqnarray}
\rho(T)&=&1,292*\frac{273,15}{273,15+ T}\label{eq:aircaracteristic1}\\
F_d(V)&=&-\frac{1}{2}.\rho(T).Cd.S.V^2(t)\label{eq:aircaracteristic2}
\end{eqnarray}\myequations{Air Force Resistance}

where:
\begin{eqnarray*}\label{eq:aircaracteristiccst}
\rho(T)\ &=&Rho\ depending\ on\ temperature\ T\\
Cd\ &=& Drag\ Coefficient,\ depending\ on\ the\ object\\
F_d(V)\ &=&Drag\ Force\ due\ to\ Air\ Resistance\ depending\ on\ velocity\ V
\end{eqnarray*}

\subsection{Ball's characteristics in the air}

Considering those characteristics of a dimpled ball model,

\begin{eqnarray*}\label{eq:massdiameterintertia}
mass\ m&=&0,04593\ g \\
Dynamic\ viscosity\ \mu&=&1,5.10^{-6} m^2/s\\
Diameter\ D&=&0,04267\ m\\
Radius\ r&=&\frac{D}{2}\\
Inertia\ I&=&\frac{2}{5}.m.R^2\\
Cross-section\  S&=&\pi.R^2\\
Ball\ Velocity\ &=&V
\end{eqnarray*}\myequations{Ball Mass, Diameter, Rayon, Intertia, Section, Reyolds Number}

We have an estimated models,

\begin{eqnarray}\label{eq:dragforce}
Re&=&\frac{\rho(T).V.D}{\mu}\\
Cd&=&0,36 + \frac{24}{Re} + \frac{6}{1+\sqrt{Re}}\\
\overrightarrow{F_{d}} &=& -\frac{1}{2}.\rho(T).Cd.S.\sqrt{(V(x)^2+V(y)^2+V(z)^2)}.\widehat{V}(x)
\end{eqnarray}\myequations{Drag Coefficient, Drag Force}

where,

\begin{eqnarray*}\label{eq:dragforce}
Re&=&Reynolds\ Number\ estimator\\
Cd&=&Drag\ Coefficient\ White's\ relation\\
\overrightarrow{F_{d}}&=& Drag\ Force
\end{eqnarray*}


\myremarks{Finally, the equation of the drag Force along the 3-axes becomes \ref{eq:vectordragforce}}{\begin{equation}\label{eq:vectordragforce}
\overrightarrow{F_d} = \begin{pmatrix}
-\frac{1}{2}.\rho(T).Cd.S.V_{x}^2 \\
-\frac{1}{2}.\rho(T).Cd.S.V_{y}^2 \\
-\frac{1}{2}.\rho(T).Cd.S.V_{z}^2
\end{pmatrix}
\end{equation}
}

\section{Magnus Force}
The main bibliography is \cite{GPA05} et \cite{SASDR94}. The Magnus Force depends on the Lift Coefficient. I have found several ways to calculate the Cl. given in \ref{eq:magnusforce4},  \ref{eq:magnusforce6} or \ref{eq:magnusforce6}. 

Given severals Magnus Force equations bellow, found in litterature,
 
\begin{eqnarray}\label{eq:magnusforce}
\overrightarrow{F_{M}} &=& Cm.(\overrightarrow{\omega}\ x\  \overrightarrow{V})\label{eq:magnusforce1}\\
F_M&=& 1/2.Cm.\rho.\mu^2\label{eq:magnusforce2}\\
Cm&=&\frac{r\omega}{V}\label{eq:magnusforce3}\\
Cm&=&-0,05+ \sqrt{0,0025 + 0,36\left( \frac{r\omega}{V_m}\right)}\ (\cite{GPA05})\label{eq:magnusforce4}
\end{eqnarray}\myequations{Magnus Lift Force}
where,
\begin{eqnarray*}\label{eq:dragforce}
\omega&=&spin\ in\ rad/s\\
V_p&=&r\omega\ periferal\ velocity\\
V_m&=&Ball\  Velocity
\end{eqnarray*}


And, espacially in \cite{SASDR94} or \cite{AL98}, an expression of Magnus Lift Coefficient.
\begin{eqnarray}
Cl&=&\frac{1}{2 + (v/R*sin(\zeta))}\  \zeta\ to\ be\ the\ angle\ between\ the\ axis\ of\ rotation\ and\ the\ direction\ of\ motion\\\label{eq:magnusforce6}
Cm &=&0,5 . \rho(T). S. R^2. Cl.\mid{\frac{\omega}{Vm}}\mid^{0,4}.Vm\ (\cite{SASDR94})\label{eq:magnusforce5}
\end{eqnarray}\myequations{Magnus Lift Force}

In order to meet my needs, I have chosen to provide adapted Cl values to (\ref{eq:magnusforce6}) from experience with my own balls (texture and number dipples). These data are recorded in a Scilab table $Cl_1$ :
\begin{eqnarray}
Cm &=&0,5 . \rho(T). S. R^2. Cl_1.\mid{\frac{\omega}{Vm}}\mid^{0,4}.Vm\ \label{eq:magnusforce7}
\end{eqnarray}\myequations{Magnus Lift Force}

\begin{eqnarray*}
with\ my\ own\ \  Cl_1(Club\ , myball)&=&['D'      ;'W5'     ;'H3'    ;'5'      ;'6'     ;'7'     ;'8'      ;'9'    ;'PW'     ;'AW'    ;'SW'      ;'LW'   ;'PU'   ] \\
&&[0.64     ;0.65     ;0.54    ;0.54     ;0.83    ;0.42    ;0.53     ;0.53   ;0.52     ;0.52    ;0.51      ;0.51   ;0]
\end{eqnarray*}\myequations{My Cl1}

\myremarks{Knowing \ref{eq:magnusforce1} and \ref{eq:magnusforce7}, we have the following Magnus force vector}{\begin{equation}\label{eq:vectormagnusforce}
\overrightarrow{F_m} = \begin{pmatrix}
Cm*(\omega_j*v_k-\omega_k*v_j) \\
Cm*(\omega_k*v_i-\omega_i*v_k) \\
Cm*(\omega_i*v_j-\omega_j*v_i)
\end{pmatrix}
\end{equation}
}

\section{Gravity Force}
\section{Launch Velocity}
\cite{PA01}
\section{Complete Model of flight}

\section{Ball's characteristics bouncing on the grass}
\cite{rwlc2010}
\section{Ball's characteristics rolling on the grass}
\cite{PA07}
\cite{PA022}


\chapter{Golf Ball Capturing}
\section{section}
section
\subsection{subsection}
subsection

\chapter{Listings}
\section{Golfball.sci}
\begin{lstlisting}[language=matlab, caption=Listing of the golf ball fligth, label={lst:gaolfballsci}]
// Copyright (C) 2016 - Corporation - Author
//
// About your license if you have any
//
// Date of creation: 16 juin 2016
//
//SCI2C: DEFAULT_PRECISION= DOUBLE
//
// ReadWave : FFT_Mat.sce
// 

\end{lstlisting}

\bibliographystyle{alpha}
\selectlanguage{english}
	\bibliography{biblioBibTex}
%\end{multicols}\newpage



%\input{PageEnd.tex}
\end{document}
